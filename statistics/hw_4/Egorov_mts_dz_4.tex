% !TeX spellcheck = russian-aot
\documentclass[12pt, a4paper]{article}
\usepackage[utf8]{inputenc}
\usepackage[T2A]{fontenc}
\usepackage[russian]{babel}
\usepackage[oglav,spisok,boldsect,eqwhole,figwhole,hyperref,hyperprint,remarks,greekit]{fn2kursstyle}


\usepackage{subfig}
\usepackage{amsfonts}
\usepackage{mathtools}
\usepackage{enumitem}
\usepackage{pifont}
\usepackage{changepage}
\usepackage{multirow}
\usepackage{supertabular}
\usepackage{multicol}
\usepackage{scalerel}
\usepackage{graphicx}

\usepackage{colortbl}
\usepackage{nicematrix}
\usepackage{stackengine}

\frenchspacing
\sloppy
\counterwithout{equation}{section}
\counterwithout{figure}{section}
\newenvironment{comment}{}{}

\graphicspath{
	{./style/},
	{./illustr/}
}

% Переопределение команды \vec, чтобы векторы печатались полужирным курсивом
\renewcommand{\vec}[1]{\text{\mathversion{bold}${#1}$}}%{\bi{#1}}

\renewcommand{\phi}{\varphi}
\newcommand\TTa{\scalerel*{\tTa}{T}}

\newcommand\thh[1]{\text{\mathversion{bold}${#1}$}}
\newcommand\xrowht[2][0]{\addstackgap[.5\dimexpr#2\relax]{\vphantom{#1}}}
%Переопределение команды нумерации перечней: точки заменяются на скобки
%\renewcommand{\labelenumi}{\theenumi)}
\renewcommand{\labelenumii}{\arabic{enumi}.\arabic{enumii}}
\renewcommand{\labelenumiii}{\arabic{enumi}.\arabic{enumii}.\arabic{enumiii}}
\renewcommand{\labelenumiv}{\arabic{enumi}.\arabic{enumii}.\arabic{enumiii}.\arabic{enumiv}}



\subject{}
\worktype{Домашнее задание}
\title{Математика --- Домашнее задание 4}
\variant{А.\,Д.~Егоров}	
\group{}
\author{А.\,Д.~Егоров}
\date{2023}

\begin{document}
	
	\maketitle
	\tableofcontents
	\pagebreak
	
	\section{Задача №1}
	
		\subsection*{Условие}		
			Правильная монетка подбрасывается до тех пор пока не выпадет $n$ орлов подряд. Найти математическое ожидание необходимого числа бросков.
		\subsection*{Решение}
		
			Пусть $\xi_k$ --- число бросков необходимых для $k$ орлов подряд, $p$ --- вероятность выпадения орла, тогда запишем следующую формулу:
			\begin{equation}
				\label{Exi_k}
				E\xi_k = p (E\xi_{k-1} + 1) + \left(1 - p\right) 
				\left( E\xi_{k-1} + 1 + E\xi_k\right).
			\end{equation}
			\looser{-0.02}{Рассмотрим данную формулу: для того, чтобы получить $k$ орлов, необходимо до этого выбить $k - 1$ орла и сделать еще один бросок. Отсюда в формуле появляется составляющая $E\xi_{k-1} + 1$. Логично, что при броске, следующим за $k - 1$, может выпасть решка, тогда придется начинать процесс выбивания орла заново (отсюда прибавка $E\xi_k$ в третьей скобке). Вероятность успешного исхода равна $p$ (вероятности выпадения орла), вероятность провала равна $1 - p$, соответственно.} 
			Выразим $E\xi_k$ через $E\xi_{k-1}$:
			$$
				E\xi_k = \dfrac{1}{p}(E\xi_{k-1} + 1).
			$$
			Получилась рекурсивная формула. Подставим в нее значения $k = \overline{1, n}$:
			\begin{gather*}
				E\xi_1 = \dfrac{1}{p}(E\xi_{0} + 1) = \dfrac{1}{p},
			\end{gather*}
			т.~к. мат. ожидание количества бросков для получения 0 орлов подряд равно 0,
			\begin{gather*}
				E\xi_2 = \dfrac{1}{p}(E\xi_{1} + 1) = \dfrac{1}{p^2} + \dfrac{1}{p}, \\
				E\xi_3 = \dfrac{1}{p}(E\xi_{2} + 1) = \dfrac{1}{p^3} + \dfrac{1}{p^2} + \dfrac{1}{p}, \\
				\cdots, \\
				E\xi_n = \dfrac{1}{p}(E\xi_{n-1} + 1) = \dfrac{1}{p^n} + \dfrac{1}{p^{n-1}} + \cdots + \dfrac{1}{p^2} + \dfrac{1}{p}. \\
			\end{gather*}
			Видно, что получается сумма геометрической прогрессии, и, так как монетка правильная, т.~е. $p = \dfrac{1}{2}$, получим 
			$$
				E\xi_n = \sum_{k=1}^{n} \dfrac{1}{p^k} = \dfrac{\frac{1}{p} (\frac{1}{p^n} -1)}{\frac{1}{p} - 1} = 2 (2^n - 1).
			$$
			 
		
	\newpage
	\section{Задача №2}
	
		\subsection*{Условие}	
			Авария происходит в точке $\mathrm{X}$, которая равномерно распределена на дороге длиной $\mathrm{L}$. Во время аварии машина скорой помощи находится в точке $\mathrm{Y}$, которая так же равномерно распределена на дороге. Считая, что $\mathrm{X}$ и $\mathrm{Y}$ независимы, найти математическое ожидание расстояния между машиной скорой помощи и точкой аварии.
		\subsection*{Решение}
			$X$, $Y$ --- случайные величины, отвечающие за распределения точек $\mathrm{X}$, $\mathrm{Y}$.
			Из условия:
			$X \sim \mathrm{R} \left[0, L\right]$, $Y  \sim \mathrm{R} \left[0, L\right]$, т.~е. функции плотности распределения $X, Y$ следующие: $f_{X} = f_{Y} = \dfrac{1}{L}$. Так как сказано, что $X$ и $Y$ являются независимыми случайными величинами, то верна следующая формула для функции плотности совместного распределения $X$, $Y$:
			$$
				f_{XY} = f_X f_Y = \dfrac{1}{L^2}
			$$
			Нас интересует расстояние между точками $|X - Y|$, а точнее его математическое ожидание $E|X - Y|$, найдем его:
			$$
				E|X - Y| = \int_{0}^{L} \int_{0}^{L} |x - y| \dfrac{1}{L^2} \ \mathrm{d} y \ \mathrm{d} x.
			$$
			Вычислим интегралы:
			\begin{gather*}
					\int_{0}^{L} |x - y| \ \mathrm{d} y = 
					\int_{0}^{x} (x - y) \ \mathrm{d} y +
					\int_{x}^{L} (y - x) \ \mathrm{d} y = 	
					\dfrac{1}{L^2} - L x + x^2, \\
					\int_{0}^{L} \left( \dfrac{1}{L^2} - L x + x^2 \right) \ \mathrm{d} x = \dfrac{L^3}{3}.
			\end{gather*}
			Подставим и получим 
			$$
				E|X - Y| = \dfrac{1}{L^2} \dfrac{L^3}{3} = \dfrac{L}{3}.
			$$
			
	\newpage
	\section{Задача №3}
	
		\subsection*{Условие}
			 \vspace*{-1mm}
			 \begin{table}[!h]
				 	\centering
				 	\caption{\label{table_XY} Совместный закон распределения случайных величин $X$ и $Y$}
				 	\vspace*{2mm}
				 	%\hspace*{-8mm}
				 	\begin{NiceTabular}{|c|c|c|c|}[colortbl-like]
					 		
					 		\hline
					 		$X \backslash Y$
					 		& 0
					 		& 1
					 		& 3 \\
					 			
					 		\hline
					 		0
					 		& 0.15
					 		& 0.05
					 		& 0.3\\
					 		
					 		\hline
					 		-1
					 		& 0
					 		& 0.15
					 		& 0.1\\
					 		
					 		\hline
					 		-2
					 		& 0.15
					 		& 0
					 		& 0.1\\
					
					 		\hline
					 	\end{NiceTabular}
				 	\label{table: XY}				
				 \end{table}		
			 \vspace*{2mm}
			Найти: 
			\begin{itemize}
				\item законы распределения случайных величин $\mathrm{X}$, $\mathrm{Y}$,
				\item EX, EY, DX, DY, cov(X, Y), corr(X, Y) а также EV, DV, где $\mathrm{V} = 6\mathrm{X} - 4 \mathrm{Y} + 3$.
			\end{itemize}
		\subsection*{Решение}
		
			Для нахождения частного закона распределения случайной величины $X$ просуммируем значения в строках, а для $Y$ суммируем значения в столбцах.
			\vspace*{-1mm}
			\begin{table}[!h]
				\centering
				\caption{\label{table_X} Закон распределения случайной величины $X$}
				\vspace*{2mm}
				%\hspace*{-8mm}
				\begin{NiceTabular}{|c|c|c|c|}[colortbl-like]
					
					\hline
					$X$
					& 0
					& -1
					& -2 \\
					
					\hline
					$P(X)$
					& 0.5
					& 0.25
					& 0.25\\
					
					\hline
				\end{NiceTabular}
				\label{table: X}				
			\end{table}		
			\vspace*{2mm}
			
			\vspace*{-1mm}
			\begin{table}[!h]
				\centering
				\caption{\label{table_Y} Закон распределения случайной величины $Y$}
				\vspace*{2mm}
				%\hspace*{-8mm}
				\begin{NiceTabular}{|c|c|c|c|}[colortbl-like]
					
					\hline
					$Y$
					& 0
					& 1
					& 3 \\
					
					\hline
					$P(Y)$
					& 0.3
					& 0.2
					& 0.5\\
					
					\hline
				\end{NiceTabular}
				\label{table: Y}				
			\end{table}		
			\vspace*{2mm}
			
			\looser{-0.03}{Для нахождения мат. ожидания и дисперсии воспользуемся следующими формулами:}
			$$
				E\xi = \sum_{i=0}^{n} p(\xi_i) \xi_i, 
				\quad 
				D\xi = E\xi^2 - (E\xi)^2 = \sum_{i=0}^{n} p(\xi_i) \xi_i^2 - \left( \sum_{i=0}^{n} p(\xi_i) \xi_i \right)^2,
			$$
			где $\xi$ --- случайная величина, $\xi_i$ --- значение этой случайной величины. Для случайных величин $X$ и $Y$ получим
			\begin{gather*}
				EX = -1 \cdot 0.25 - 2 \cdot 0.25 = - 0.75,
				\quad DX = 1 \cdot 0.25 + 4 \cdot 0.25 - (0.75)^2 = 0.6875, \\
				EY = 1.7, \quad DY = 1.81.
			\end{gather*}
			Найдем ковариацию и коэффициент корреляции: 
			\begin{gather*}
				cov(X, Y) = E(XY)- (EX) (EY),
				\quad
				corr(X, Y) = \dfrac{cov(X, Y)}{\sqrt{DX} \sqrt{DY}}, \\
				EXY = \sum_{i=0}^{n} \sum_{j=0}^{m} p_{ij} X_i Y_j  = 
				1 \cdot (-1) \cdot 0.15 + 3 \cdot (-1) \cdot 0.1 + 3 \cdot (-2) \cdot 0.1 = -1.05, \\
				cov(X, Y) = -1.05 - (-0.75) \cdot 1.7 = 0.225, \\
				corr(X, Y) = \dfrac{0.225}{\sqrt{0.6875} \sqrt{1.81}} =  0.2017.
			\end{gather*}
			Теперь найдем математическое ожидание и дисперсию случайной величины $V$: 
			\begin{gather*}
				EV = E(6X - 4Y + 3) = 6EX -4EY + 3 = -8.3,\\
				DV = D(6X - 4Y + 3) = D(6X - 4Y) = D(6X) + D(4Y) - 2 cov(6X, 4Y) = \\
				= 36 D(X) + 16 D(Y) - 48 cov(X, Y) = 
				36 \cdot 0.6875 + 16 \cdot 1.81 - 48 \cdot 0.225 = 42.91.
			\end{gather*}
			
				
			
	\newpage
	\section{Задача №4}
	
		\subsection*{Условие}
			Пусть $x_1, x_2, \dots, x_n$ --- результаты $n$ независимых повторных наблюдений над дискретной случайной величиной $\xi$, принимающей значения из множества $\mathrm{Y} = \{ 0, 1\}$ с вероятностями: 
			$$
				P\left(\xi = 0\right) = \dfrac{1 - \theta}{2}, 
				\quad
				P\left(\xi = 1\right) = \dfrac{1 + \theta}{2}, 
				\quad
				-1 < \theta < 1.
			$$
			Найти оценку максимального правдоподобия (МП) для $\theta$.
			
		\subsection*{Решение}
			Имеем закон распределения $Bern(\frac{1 - \theta}{2})$,			
			c учетом, что было проведено $n$ испытаний, функция максимального правдоподобия будет следующая: 
			$$
				L(\underline{x}, \theta) = \left(\dfrac{1 - \theta}{2}\right)^{\sum_{j = 1}^{n} x_j} 
				\left(\dfrac{1 - \theta}{2}\right)^{n - \sum_{j = 1}^{n} x_j}
				%\prod_{j = 1}^{n}  C_n^{x_j} \left(\dfrac{1 - \theta}{2}\right)^{x_j} \left(\dfrac{1 + \theta}{2}\right)^{n - x_j},
			$$
			прологарифмируем и получим
			$$
				\ln L(\underline{x}, \theta) = \sum_{j = 1}^{n} x_j \ln{\left(\dfrac{1 - \theta}{2}\right)} + 
				(n - \sum_{j = 1}^{n} x_j) \ln{\left(\dfrac{1 - \theta}{2}\right)}.
			$$
			Продифференцируем по $\theta$ и приравняем к нулю:
			\begin{gather*}
				\dfrac{ \partial \ln L(\underline{x}, \theta) }{ \partial \theta} 
				= 
				- \dfrac{1}{2} \sum_{j = 1}^{n} x_j \left( \dfrac{2}{1 - \theta} \right) +
				\dfrac{1}{2} (n - \sum_{j = 1}^{n} x_j) \left( \dfrac{2}{1 + \theta} \right)	= 0, 
				%n \overline{x} (1 + \theta) + n (1 - \theta) - n \overline{x} (1 - \theta) = 0, \\
			\end{gather*}
			отсюда получим, что $\theta_{ML} = 1 - 2 \overline{x}$.
			
	\newpage
	\section{Задача №5}
	
		\subsection*{Условие}	
			Пусть $x_1, x_2, \dots, x_n$ --- результаты $n$ независимых повторных наблюдений над случайной величиной $\xi$, плотность распределения которой имеет вид
			\begin{gather*}
				f(x,\,\theta) = 
				p f_1(x,\,\theta) + (1 - p) f_2(x,\,\theta), \\
				f_1(x,\,\theta) = 
				\begin{cases}
					\dfrac{1}{\theta}, &0 < x \leqslant \theta, \\
					0, &else,
				\end{cases} 
				\quad 
				f_2(x,\,\theta) = 
				\begin{cases}
					\dfrac{1}{1 - \theta}, &\theta < x \leqslant 1, \\
					0, &else,
				\end{cases}
			\end{gather*}
			Найти оценки неизвестных параметров $p$, $\theta$ методом моментов.
		\subsection*{Решение}
			Найдем истинные моменты первого и второго порядка: 
			\begin{gather*}
				X^1 = \int_{0}^{1} f(x, \theta) \ \mathrm{d} x = 
				p \int_{0}^{\theta} f_1(x, \theta) \ \mathrm{d} x + 
				(1 - p) \int_{\theta}^{1} f_2(x, \theta) \ \mathrm{d} x = 
				\dfrac{1}{2} (1 - p + \theta),\\
				X^2 = \int_{0}^{1} x^2 f(x, \theta) \ \mathrm{d} x = 
				\dfrac{1}{3} (1 + \theta + \theta^2 - p ( 1 + \theta)),
			\end{gather*}
			приравняем их к соответствующим им выборочным моментам: 
			\begin{gather*}
				\begin{cases}
					\dfrac{1}{2} (1 - p + \theta) = \overline{X},\\
					\dfrac{1}{3} (1 + \theta + \theta^2 - p ( 1 + \theta)) = \overline{X^2},\phantom{\vline height 2em}
				\end{cases}
			\end{gather*}
			--- система из 2-х уравнений и 2-х неизвестных, решив ее, получим ответ: 
			$$
				p = \dfrac{1-2\overline{X} + 4 \overline{X}^2 - 3 \overline{X^2}}{1 - 2 \overline{X}}, 
				\quad
				\theta = \dfrac{2 \overline{X} - 3 \overline{X^2}}{1 - 2 \overline{X}}.
			$$
			
			
	
\end{document}
