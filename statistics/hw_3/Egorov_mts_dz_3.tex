% !TeX spellcheck = russian-aot
\documentclass[12pt, a4paper]{article}
\usepackage[utf8]{inputenc}
\usepackage[T2A]{fontenc}
\usepackage[russian]{babel}
\usepackage[oglav,spisok,boldsect,eqwhole,figwhole,hyperref,hyperprint,remarks,greekit]{fn2kursstyle}


\usepackage{subfig}
\usepackage{amsfonts}
\usepackage{mathtools}
\usepackage{enumitem}
\usepackage{pifont}
\usepackage{changepage}
\usepackage{multirow}
\usepackage{supertabular}
\usepackage{multicol}
\usepackage{scalerel}
\usepackage{graphicx}

\usepackage{colortbl}
\usepackage{nicematrix}
\usepackage{stackengine}

\frenchspacing
\sloppy
\counterwithout{equation}{section}
\counterwithout{figure}{section}
\newenvironment{comment}{}{}

\graphicspath{
	{./style/},
	{./illustr/}
}

% Переопределение команды \vec, чтобы векторы печатались полужирным курсивом
\renewcommand{\vec}[1]{\text{\mathversion{bold}${#1}$}}%{\bi{#1}}

\renewcommand{\phi}{\varphi}
\newcommand\TTa{\scalerel*{\tTa}{T}}

\newcommand\thh[1]{\text{\mathversion{bold}${#1}$}}
\newcommand\xrowht[2][0]{\addstackgap[.5\dimexpr#2\relax]{\vphantom{#1}}}
%Переопределение команды нумерации перечней: точки заменяются на скобки
%\renewcommand{\labelenumi}{\theenumi)}
\renewcommand{\labelenumii}{\arabic{enumi}.\arabic{enumii}}
\renewcommand{\labelenumiii}{\arabic{enumi}.\arabic{enumii}.\arabic{enumiii}}
\renewcommand{\labelenumiv}{\arabic{enumi}.\arabic{enumii}.\arabic{enumiii}.\arabic{enumiv}}



\subject{}
\worktype{Домашнее задание}
\title{Математика --- Домашнее задание 3}
\variant{А.\,Д.~Егоров}	
\group{}
\author{А.\,Д.~Егоров}
\date{2023}

\begin{document}
	
	\maketitle
	\tableofcontents
	\pagebreak
	
	\section{Задача №1}
	
		\subsection*{Условие}		
			Компания по страхованию автомобилей разделяет водителей по 3 классам: класс А (мало
			рискует), класс В (рискует средне), класс С (рискует сильно). Компания предполагает, что
			из всех водителей, застрахованных у неё, 30\% принадлежат классу А, 50\% – классу В, 20\% –
			классу С. Вероятность того, что в течение года водитель класса А попадёт хотя бы в одну
			автокатастрофу, равна 0,01; для водителя класса В эта вероятность равна 0,03, а для
			водителя класса С – 0,1. Мистер Джонс страхует свою машину у этой компании и в течение
			года попадает в автокатастрофу. Какова вероятность того, что он относится к классу А?
		
			
		\subsection*{Решение}
		
			\looser{0.02}{Пусть $
			H = \{\text{человек страхует машину и в течении года попадет в ДТП}\}
			$, а }
			$
				H_i = \{\text{человек из } i\text{-ого класса страхует машину и в течении года попадет в ДТП}\}.
			$ 
			Получаем, что 
			$
				P(H) = P(H_A) + P(H_B)  + P(H_C) = 0.3 \cdot 0.01 + 0.5 \cdot 0.03 + 0.2 \cdot 0.1 = 0.038.
			$
			Тогда вероятность, что человек принадлежит классу А, при условии, что он попал в аварию, следующая:
			$$
				\dfrac{P(H_A)}{P(H)} = \dfrac{0.003}{0.038} \approx 0.0789.
			$$
			
	\newpage
	\section{Задача №2}
	
		\subsection*{Условие}		
			Движением частицы по целым точкам прямой управляет схема Бернулли с вероятностью
			$р$ исхода $1$. Если в данном испытании схемы Бернулли появилась $1$, то частица из своего
			положения переходит в правую соседнюю точку, а в противном случае - в левую.
			Найти	
			вероятность того, что за $n$ шагов частица из точки $0$ перейдет в точку $m$.
		
		\subsection*{Решение}
			Положим, что для того, чтобы дойти до точки $m$ потребовалось сделать 
			$k$ шагов вправо и, соответственно, $n - k$ шагов влево. Также допустим, что точка $m$ находится справа от точки $0$, т.~е. $k \geqslant n - k$, тогда $m = k - (n - k)$ или $k = \dfrac{n + m}{2}$. Тогда для того, чтобы найти вероятность попадания из точки 0 в точку $m$ за $n$ шагов, воспользуемся формулой вероятности для биномиального закона $Bi(n,p)$:
			$$
				P(\text{\{из т. 0 в т. $m$ за $n$ шагов\}}) = C_n^{k}\, p^k (1- p)^{n-k} = 
				%C_{n}^{\frac{n+m}{2}} p^{\frac{n+m}{2}} (1- p)^{n-\frac{n+m}{2}} = 
				C_{n}^{\frac{n+m}{2}}\, p^{\frac{n+m}{2}} (1- p)^{\frac{n-m}{2}}.
			$$
			Аналогично для случая, когда точка $m$ слева от $0$:
			$$
			P(\text{\{из т. 0 в т. $m$ за $n$ шагов\}}) = 
			C_{n}^{\frac{n-m}{2}}\, p^{\frac{n-m}{2}} (1- p)^{\frac{n+m}{2}}.
			$$
			
	\newpage
	\section{Задача №3}
	
		\subsection*{Условие}
			Плотность распределения $p(x)$ некоторой случайной величины имеет вид
			$$ p(x) = \dfrac{C}{\mathrm{e}^x + \mathrm{e}^{-x}}$$
			где $\mathrm{C}$ – константа. Найти значение этой константы c и вероятность того, что случайная
			величина примет значение, принадлежащее интервалу $\left(-\pi, \pi\right)$.		
			
		\subsection*{Решение}
			Для нахождения константы воспользуемся условием нормировки: 
			\begin{gather*}
				\int_{-\infty}^{\infty} p(x) \mathrm{d} x = 
				\int_{-\infty}^{\infty} \dfrac{C}{\mathrm{e}^x + \mathrm{e}^{-x}} \mathrm{d} x = 1,
			\end{gather*}
			отсюда получим, что $C = \dfrac{2}{\pi}$. Тогда вероятность того, что случайная величина примет значение, принадлежащее интервалу $\left(-\pi, \pi\right)$, будет следующая:
			\begin{gather*}
				\int_{-\pi}^{\pi} \dfrac{2 / \pi}{\mathrm{e}^x + \mathrm{e}^{-x}} \mathrm{d} x = \dfrac{2}{\pi} 
				\left( 
				\arctg{\left(\mathrm{e}^{\pi}\right)}
				-\arctg{\left(\mathrm{e}^{-\pi}\right)} 
				\right)
				\approx 0.945013.
			\end{gather*}
			
			
			
		
	\newpage
	\section{Задача №4}
	
		\subsection*{Условие}
		Задана плотность распределения вероятностей $f(x)$ непрерывной случайной величины $X$:
		$$
			f(x) = 
			\begin{cases}
				A \sqrt{x}, & x \in [1, 4], \\
				0, & x \not \in [1, 4].
			\end{cases}
		$$		
		Найти функцию распределения $F(x)$ и $P(2 < X < 3)$.		
		
		\subsection*{Решение}
		
		Из условия нормировки найдем константу A: 
		\begin{gather*}
			\int_{-\infty}^{\infty} f(x) \mathrm{d} x = 
			\int_{1}^{4} A \sqrt{x} \, \mathrm{d} x = 1,
		\end{gather*}
		отсюда $A = \dfrac{3}{14}$. Тогда функция распределения с.~в. $X$ имеет вид
		\begin{gather*}
			F(x) = \int_{-\infty}^{x} f(y) \, \mathrm{d} y = 
			\int_{1}^{x} \dfrac{3}{14} \sqrt{y} \, \mathrm{d} y = 
			\dfrac{1}{7} \left(x^{3/2} - 1\right),
		\end{gather*}
		а вероятность $P(2 < X < 3) = F(3) - F(2) = \dfrac{1}{7} \left(3^{3/2} - 2^{3/2}\right)$.
		
		
	
	\newpage
	\section{Задача №5}
	
		\subsection*{Условие}	
		При работе ЭВМ время от времени возникают сбои. Поток сбоев можно считать
		простейшим.		
		Среднее число сбоев за сутки равно $1.5$.
		Найти вероятность того, что в течение суток произойдет хотя бы один сбой.	
				
		\subsection*{Решение}
		
		Число сбоев $\lambda = 1.5$, поток простейший (пуассоновский случайный процесс), тогда вероятность события $A = $ \{в течении суток произойдет хотя бы 1 сбой\}, будет следующая: 
		\begin{gather*}
			P(A) = P(X \geqslant 1) = 1 - P(X = 0) = 1 - \mathrm{e}^{-\lambda}
			= 1 - \mathrm{e}^{-1.5} \approx 0.77687,
		\end{gather*}
		где $X$ --- случайная величина, отвечающая за число сбоев.

		
		
		
	
\end{document}