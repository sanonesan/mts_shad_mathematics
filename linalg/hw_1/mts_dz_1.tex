% !TeX spellcheck = russian-aot
\documentclass[12pt, a4paper]{article}
\usepackage[utf8]{inputenc}
\usepackage[T2A]{fontenc}
\usepackage[russian]{babel}
\usepackage[oglav,spisok,boldsect,eqwhole,figwhole,hyperref,hyperprint,remarks,greekit]{fn2kursstyle}


\usepackage{subfig}
\usepackage{amsfonts}
\usepackage{mathtools}
\usepackage{enumitem}
\usepackage{pifont}
\usepackage{changepage}
\usepackage{multirow}
\usepackage{supertabular}
\usepackage{multicol}
\usepackage{scalerel}
\usepackage{graphicx}

\usepackage{colortbl}
\usepackage{nicematrix}
\usepackage{stackengine}

\frenchspacing
\sloppy
\counterwithout{equation}{section}
\counterwithout{figure}{section}
\newenvironment{comment}{}{}

\graphicspath{
	{./style/},
	{./illustr/}
}

% Переопределение команды \vec, чтобы векторы печатались полужирным курсивом
\renewcommand{\vec}[1]{\text{\mathversion{bold}${#1}$}}%{\bi{#1}}

\renewcommand{\phi}{\varphi}
\newcommand\TTa{\scalerel*{\tTa}{T}}

\newcommand\thh[1]{\text{\mathversion{bold}${#1}$}}
\newcommand\xrowht[2][0]{\addstackgap[.5\dimexpr#2\relax]{\vphantom{#1}}}
%Переопределение команды нумерации перечней: точки заменяются на скобки
%\renewcommand{\labelenumi}{\theenumi)}
\renewcommand{\labelenumii}{\arabic{enumi}.\arabic{enumii}}
\renewcommand{\labelenumiii}{\arabic{enumi}.\arabic{enumii}.\arabic{enumiii}}
\renewcommand{\labelenumiv}{\arabic{enumi}.\arabic{enumii}.\arabic{enumiii}.\arabic{enumiv}}



\subject{}
\worktype{Домашнее задание}
\title{Математика --- Домашнее задание 1}
\variant{А.\,Д.~Егоров}	
\group{}
\author{А.\,Д.~Егоров}
\date{2023}

\begin{document}
	
	\maketitle
	\tableofcontents
	\pagebreak
	
	\section{Задача №1}
	
		\subsection*{Условие}
		
			Зная, что $|\vec{a}| = 2, |\vec{b}| = 5, \angle{ \left( \vec{a}, \vec{b}\right) } = \dfrac{2\pi}{3}$. При каком коэффициенте $\alpha$ векторы $\vec{p} = \alpha \vec{a} + 17 \vec{b}$ и $\vec{q} = 3 \vec{a} - \vec{b}$ перпендикулярны.
		
		\subsection*{Решение}
			Векторы должны быть перпендикулярны, т.~е.
			$
			\angle(\vec{p}, \vec{q}) = \dfrac{\pi}{2}
			$
			или 		
			$
				\cos \angle(\vec{p}, \vec{q}) = 0.
			$
			Воспользуемся следующей формулой:
			\begin{equation*}
				\label{cos}
				\cos \angle(\vec{p}, \vec{q}) = 
				\dfrac{\left( \vec{p}, \vec{q} \right)}{|\vec{p}| \, |\vec{q}|},
				\quad
				\Rightarrow \quad ( \vec{p}, \vec{q} ) = 0.
			\end{equation*}
			Вычислим скалярное произведение $\vec{p}$ и $\vec{q}$:
			\begin{gather*}
				( \vec{p}, \vec{q} ) = 
				(\alpha \vec{a} + 17 \vec{b}, 3 \vec{a} - \vec{b}) = 
				(\alpha \vec{a}, 3 \vec{a}) + (17 \vec{b}, 3 \vec{a}) - (\alpha \vec{a}, \vec{b}) - (17 \vec{b},  \vec{b}) = \\
				= 3 \alpha |\vec{a}|^2 + (51 - \alpha) 
				|\vec a| |\vec b| \cos \angle(\vec{a}, \vec{b}) - 17 |\vec b|^2 = 0.
			\end{gather*}
			Выразим $\alpha$:
			\begin{gather*}
				\alpha (3 |\vec a|^2 - |\vec a| |\vec b| \cos \angle(\vec{a}, \vec{b}) ) =
				17 |\vec b|^2 - 51 |\vec a| |\vec b| \cos \angle(\vec{a}, \vec{b}),
				\\
				\alpha =
				\dfrac{17 |\vec b|^2 - 51 |\vec a| |\vec b| \cos \angle(\vec{a}, \vec{b})}{(3 |\vec a|^2 - |\vec a| |\vec b| \cos \angle(\vec{a}, \vec{b}) )}.
			\end{gather*}
			Подставляя числа из условия в полученную формулу для $\alpha$, получим
			$$
			\text{Ответ:} \quad \alpha = 40.
			$$
		
	\newpage
	\section{Задача №2}
	
		\subsection*{Условие}
			Множество 
			$
			\mathbf{G} = 
			\{ x: x = (x_1, x_2, \dots, x_n)^T \in \mathbb{R}^n,\ |\ x_i \geqslant 0, i=\overline{1,n} \}$. Проверить является ли $\mathbf{G}$ линейным пространством (ЛП), если операции сложения и умножения определены следующим образом: 
			\begin{equation}
				\label{sum_mul}
				x + y = 
				\left(
				\begin{matrix}
					x_1 \cdot y_1 \\
					x_2 \cdot y_2 \\
					\dots \\
					x_n \cdot y_n
				\end{matrix}
				\right), 
				\qquad
				\alpha x = 
				\left(
				\begin{matrix}
					x_1^\alpha \\
					x_2^\alpha \\
					\dots \\
					x_n^\alpha
				\end{matrix}
				\right).
			\end{equation}
			
		
		\subsection*{Решение}
			Возьмем элементы $x, y, z \in \mathbf G$. 			
			Операция суммирования \eqref{sum_mul} оставляет элемент в пространстве $\mathbf G$ 
			(произведение положительных величин --- положительная величина), 
			так и умножение \eqref{sum_mul} оставляет элемент в пространстве $\mathbf G$ 
			(возведение положительного числа в степень --- положительное число). Проверим аксиомы ЛП:
			\begin{enumerate}
				\item Коммутативность операции сложения для $\mathbf G$ очевидна.
				\item Ассоциативность операции сложения для $\mathbf G$ очевидна.
				\item Нулевой элемент ($\vec{x} + \vec{a} = \vec{x}$) --- вектор из единиц.
				\item Для любого элемента найдется противоположный ($\vec{x} + \vec{\overline{x}} = \vec{0}$) --- вектор из нулей.
				\item 
				Проверим останется ли элемент $u$, определенный как $u =  \beta (\alpha x + y)$, т.~е. являющийся линейной комбинацией элементов из $ \mathbf G$, в пространстве $\mathbf G$:
				$$
				u = \beta (\alpha x + y) =
				\begin{pmatrix}
					(x_1^{\alpha} \cdot y_1) ^{\beta} \\
					(x_2^{\alpha} \cdot y_2)^{\beta} \\
					\dots \\
					(x_n^{\alpha} \cdot y_n)^{\beta}
				\end{pmatrix}
				= 
				\begin{pmatrix}
					x_1^{\alpha \beta} \cdot y_1^{\beta} \\
					x_2^{\alpha \beta} \cdot y_2^{\beta} \\
					\dots \\
					x_n^{\alpha \beta} \cdot y_n^{\beta}
				\end{pmatrix}
				= \beta \alpha x + \beta y.
				$$
				Обозначим $u_i = (x_i^\alpha \cdot y_i )^\beta, \ i=\overline{1,n}$, рассмотрим данные элементы: 
				$$
				x_i^{\alpha \beta} \geqslant 0, y_i^\beta \geqslant 0 \quad \Rightarrow \quad u_i \geqslant 0,
				$$
				получается, что
				$$
				u = (u_1, u_2, \dots, u_n)^T \in \mathbb{R}^n, \ u_i \geqslant 0, \ i=\overline{1,n} 
				\quad \Rightarrow \quad
				u \in \mathbf{G}.
				$$
				Линейная комбинация элементов из $\mathbf G$ не выводит элемент из простанства $\mathbf G$.
			\end{enumerate}
			 Аксиомы линейного пространства выполнены, следовательно, $\mathbf G$ --- линейное пространство.
			
		\newpage
		
		\section{Задача №3}
		
			\subsection*{Условие}
				Доказать, что для любых векторов
				$\vec x,\ \vec y,\ \vec z$ 
				и любых чисел $\alpha,\ \beta,\ \gamma$ вектора \phantom{adsga}
				$\alpha \vec x - \beta \vec y, \ \gamma \vec y - \alpha \vec z, \ \beta \vec z - \gamma \vec x$ линейно зависимы.
			\subsection*{Решение}
				\looser{-0.03}{Составим систему из данных векторов и попытаемся свести ее к единичной матрице:}
				\begin{gather*}
					\left(
					\begin{NiceMatrix}
						\alpha & -\beta & 0 \\
						0 & \gamma & -\alpha \\
						-\gamma & 0 & \beta
					\end{NiceMatrix}
					\right) \rightarrow 
					\left(
					\begin{NiceMatrix}
						1 & -\beta / \alpha & 0 \\
						0 & 1 & -\alpha / \gamma \\
						1 & 0 & - \beta / \gamma
					\end{NiceMatrix}
					\right)  \rightarrow 	
					\left(
					\begin{NiceMatrix}
						1 & -\beta / \alpha & 0 \\
						0 & 1 & -\alpha / \gamma \\
						0 & \beta / \alpha & - \beta / \gamma
					\end{NiceMatrix}
					\right)  \rightarrow \\
					\rightarrow 	
					\left(
					\begin{NiceMatrix}
						1 & -\beta / \alpha & 0 \\
						0 & 1 & -\alpha / \gamma \\
						0 & 1 & - \alpha / \gamma
					\end{NiceMatrix}
					\right)  \rightarrow 	
					\left(
					\begin{NiceMatrix}
						1 & -\beta / \alpha & 0 \\
						0 & 1 & -\alpha / \gamma \\
						0 & 0 & 0
					\end{NiceMatrix}
					\right).
				\end{gather*}
				Последняя строчка обнулилась, следовательно, для любых векторов
				$\vec x,\ \vec y,\ \vec z$ 
				и любых чисел $\alpha,\ \beta,\ \gamma$ векторы
				$\alpha \vec x - \beta \vec y, \ \gamma \vec y - \alpha \vec z, \ \beta \vec z - \gamma \vec x$ линейно зависимы.
	
	\newpage
	\section{Задача №4}
		\subsection*{Условие}
			Проверить является ли система векторов $\vec e_i, \ i = 1, 2, 3$ базисом в линейном пространстве $\mathbb{R}^3$, и найти координаты вектора $\vec x$ в этом базисе. По известному координатному вектору $\vec y_e$ найти вектор $\vec y$.
			\begin{gather*}
				\vec e_1 = (-2, 3, 0)^T, \ 
				\vec e_2 = (2, -3, 4)^T, \
				\vec e_3 = (-2, 0, -3)^T, \\
				\vec x = (-4, 3, 7)^T, \
				\vec y_e = (4, 4, 3)^T.
			\end{gather*}
		\subsection*{Решение}
		Составим матрицу перехода к базису $\vec e_i, \ i = 1, 2, 3$ и линейными преобразованиями попытаемся свести ее к единичной:	
		\begin{gather*}
			A = 
			\left(
			\begin{NiceMatrix}
				-2 & 2 & -2 \\
				3 & -3 & 0 \\
				0 & 4 & -3
			\end{NiceMatrix}
			\right) \rightarrow
			\left(
			\begin{NiceMatrix}
				1 & -1 & 1 \\
				1 & -1 & 0 \\
				0 & 1 & -3/4
			\end{NiceMatrix}
			\right) \rightarrow
			\left(
			\begin{NiceMatrix}
				1 & -1 & 1 \\
				0 & 1 & -3/4 \\
				0 & 0 & 1
			\end{NiceMatrix}
			\right) \rightarrow
			\left(
			\begin{NiceMatrix}
				1 & 0 &0 \\
				0 & 1 & 0 \\
				0 & 0 & 1
			\end{NiceMatrix}
			\right).
		\end{gather*}
		Следовательно, векторы $\vec e_i, \ i = 1, 2, 3$ действительно образуют базис в $\mathbb{R}^3$.
		Теперь разложим вектор $\vec x$ по базису  $\vec e_i$:
		$$
			\vec x_e = A^{-1} \cdot \vec x = \left(\frac{7}{2}, \ \frac{5}{2}, \  1\right)^T.
		$$
		Найдем вектор $\vec y$:
		$$
		\vec y = A \cdot \vec y_e = \left(-6, \ 0, \  7 \right)^T.
		$$
		
		\newpage
		\section{Задача №5}
		\subsection*{Условие}
		Найти какой-нибудь базис и размерность линейного пространства $\mathbf V$, заданного следующим образом ($\mathbf P_n$ --- пространство многочленов степени 4):
		$$
		V = \{ p(x) \in \mathbf P_4 \  | \ p(1) + p(-1) = 0 \}
		$$
		\subsection*{Решение}
		Любой многочлен $p(x) \in \mathbf P_4$ определен как
		$$
			p(x) = \alpha + \beta x + \gamma x^2 + \zeta x^3 + \theta x^4,
		$$
		где $\alpha,\ \beta,\ \gamma,\ \zeta,\ \theta$ --- независимые коэффициенты при соответствующих степенях многочлена. 
		Базис в $\mathbf P_4$ задается множеством $\{1,\ x,\ x^2,\ x^3,\ x^4\}$.
		Следовательно, размерность пространства $\mathbf P_4$ равна 5.	
			 
		Для пространства $\mathbf V$ необходимо, чтобы многочлен удовлетворял равенству:  $\ p(1) + p(-1) = 0$. Подставим $p(x)$ в уравнение и упростим: 
		\begin{gather*}
			\alpha + \beta + \gamma + \zeta + \theta + 
			\alpha - \beta + \gamma - \zeta + \theta = 0, \\
			2 \alpha + 2 \gamma + 2 \theta = 0, \\
			\alpha + \gamma + \theta = 0.
		\end{gather*}
		Получается. что один из коэффициентов $\alpha,\ \gamma,\ \theta$ определяется через 2 других. Пусть $\theta$ будет зависимым коэффициентом, т.~е.
		$$
		\theta = \theta (\alpha, \gamma) = - (\alpha + \gamma).
		$$
		\looser{-0.03}{Тогда многочлены $q(x)$, принадлежащие пространству $\mathbf V$, выглядят следующим образом:}
		\begin{equation*}
			q(x) = \alpha + \beta x + \gamma x^2 + \zeta x^3 - (\alpha + \gamma) x^4.
		\end{equation*}
		Перегруппируем относительно коэффициентов и окончательно получим
		\begin{equation}
			\label{basis_4_V}
			q(x) = \alpha (1 - x^4) + \beta x + \gamma (x^2 - x^4) + \zeta x^3.
		\end{equation}
		Из \eqref{basis_4_V} \looser{0.02}{следует, что базис $\mathbf V$ можно задать множеством 
		$\{1 - x^4,\ x,\ x^2 - x^4,\ x^3\}$}
		и размерность данного пространства равна 4.
		
		
			
	

	
\end{document}