% !TeX spellcheck = russian-aot
\documentclass[12pt, a4paper]{article}
\usepackage[utf8]{inputenc}
\usepackage[T2A]{fontenc}
\usepackage[russian]{babel}
\usepackage[oglav,spisok,boldsect,eqwhole,figwhole,hyperref,hyperprint,remarks,greekit]{fn2kursstyle}


\usepackage{subfig}
\usepackage{amsfonts}
\usepackage{mathtools}
\usepackage{enumitem}
\usepackage{pifont}
\usepackage{changepage}
\usepackage{multirow}
\usepackage{supertabular}
\usepackage{multicol}
\usepackage{scalerel}
\usepackage{graphicx}

\usepackage{colortbl}
\usepackage{nicematrix}
\usepackage{stackengine}

\frenchspacing
\sloppy
\counterwithout{equation}{section}
\counterwithout{figure}{section}
\newenvironment{comment}{}{}

\graphicspath{
	{./style/},
	{./illustr/}
}

% Переопределение команды \vec, чтобы векторы печатались полужирным курсивом
\renewcommand{\vec}[1]{\text{\mathversion{bold}${#1}$}}%{\bi{#1}}

\renewcommand{\phi}{\varphi}
\newcommand\TTa{\scalerel*{\tTa}{T}}

\newcommand\thh[1]{\text{\mathversion{bold}${#1}$}}
\newcommand\xrowht[2][0]{\addstackgap[.5\dimexpr#2\relax]{\vphantom{#1}}}
%Переопределение команды нумерации перечней: точки заменяются на скобки
%\renewcommand{\labelenumi}{\theenumi)}
\renewcommand{\labelenumii}{\arabic{enumi}.\arabic{enumii}}
\renewcommand{\labelenumiii}{\arabic{enumi}.\arabic{enumii}.\arabic{enumiii}}
\renewcommand{\labelenumiv}{\arabic{enumi}.\arabic{enumii}.\arabic{enumiii}.\arabic{enumiv}}



\subject{}
\worktype{Домашнее задание}
\title{Математика --- Домашнее задание 2}
\variant{А.\,Д.~Егоров}	
\group{}
\author{А.\,Д.~Егоров}
\date{2023}

\begin{document}
	
	\maketitle
	\tableofcontents
	\pagebreak
	
	\section{Задача №1}
	
		\subsection*{Условие}		
			Найти обратную матрицы методом присоединённой матрицы (методом алгебраических дополнений):
			\begin{equation}
				\label{mat_1}
				\vec A = 
				\begin{pNiceMatrix}
					5 & 3 & 1 \\
					1 & -3 & -2 \\
					-5 & 2 & 1					
				\end{pNiceMatrix}
			\end{equation}
			
			
		\subsection*{Решение}
			Для нахождения обратной матрицы к матрице $\vec A$ методом алгебраических дополнений воспользуемся следующей формулой:
			\begin{equation}
				\label{alg_dop_eq}
				\vec A^{-1} = \dfrac{1}{\det{\vec A}} \vec {A^*}^T,
			\end{equation}
			где $\vec A^*$ --- матрица алгебраических дополнений матрицы $\vec A$.
			Найдем определитель матрицы \eqref{mat_1}: 
			\begin{equation}
				\label{det_mat_1}
				\det \vec A = 
				\begin{vNiceMatrix}
					5 & 3 & 1 \\
					1 & -3 & -2 \\
					-5 & 2 & 1					
				\end{vNiceMatrix} = 19.
			\end{equation}
			Найдем алгебраические дополнения $A^*_{ij}$ (элементы в матрице $\vec A^*$): 
			\begin{equation}
				A^*_{ij} = (-1)^{i + j} \cdot \det {\vec {\widehat A}}, \quad i,j = \overline{1,n},
			\end{equation}
			где n --- размер матрицы $\vec A$, $\vec {\widehat A}$ --- матрица, полученная из матрицы $\vec A$ исключением $i$-ой строки и $j$-ого столбца. Считая, по формуле получим:
			\begin{equation*} 
				\begin{NiceMatrix}
					A^*_{11} =\phantom{-}1, & A^*_{12} = 9, & A^*_{13} = -13, \\
					A^*_{21} = -1, & A^*_{22} = 10, & A^*_{23} = -25, \\
					A^*_{31} = -3, & A^*_{32} = 11, & A^*_{33} = -18, \\
				\end{NiceMatrix}
			\end{equation*}
			следовательно,
			\begin{equation}
				\label{alg_dop_mat}
				\vec A^* = 
				\begin{pNiceMatrix}
					\phantom{-}1 & 9 & -13 \\
					-1 & 10 & -25 \\
					-3 & 11 & -18 \\
				\end{pNiceMatrix}.
			\end{equation}
			Подставим \eqref{det_mat_1}, \eqref{alg_dop_mat} в \eqref{alg_dop_eq} и получим ответ: 
			\begin{equation*}
				\vec A^{-1} = \dfrac{1}{19}	
				\begin{pNiceMatrix}
					\phantom{-}1 & -1 & -3 \\
					9 & 10 & 11 \\
					-13 & -25 & -18 \\
				\end{pNiceMatrix}.
			\end{equation*}
			
			
		
	\newpage
	\section{Задача №2}
	
		\subsection*{Условие}
		Решить матричное уравнение:
		\begin{equation}
			\label{task_2}
			\vec A \vec X \vec B = \vec B,
		\end{equation}		
		\begin{equation*}
			\vec A = 
			\begin{pNiceMatrix}
				1 & -2 & 3 \\
				2 & 3 & -1 \\
				0 & -2 & -1 \\
			\end{pNiceMatrix},
			\quad
			\vec B = 
			\begin{pNiceMatrix}
				1 & 2 & 3 \\
				4 & 5 & 6 \\
				7 & 8 & 0 \\
			\end{pNiceMatrix}.			
		\end{equation*}
				
		\subsection*{Решение}
		Из \eqref{task_2} видно, что для того, чтобы выполнялось равенство, должно быть верно следующее:
		\begin{equation*}
			\vec A \vec X \vec B = \vec E \vec B = \vec B,
		\end{equation*}
		где $\vec E$ --- единичная матрица. Следовательно, $\vec A \vec X = \vec E$, т.~е.
		\begin{gather*}
			\vec X = \vec A^{-1}. 						
		\end{gather*}
		Находим обратную матрицу к $\vec A$ и получаем ответ:
		\begin{equation*}
			\vec X = 
			- \dfrac{1}{21}
			\begin{pNiceMatrix}
				-5 & -8 & -7 \\
				2 & -1 & 7 \\
				-4 & 2 & 7 \\
			\end{pNiceMatrix}.
		\end{equation*}
			
		\newpage
		
		\section{Задача №3}
		
			\subsection*{Условие}
			Найти ранг матрицы $\vec A$ при различных значениях параметра $\lambda$.
			\begin{equation*}
				\vec A = 
				\begin{pNiceMatrix}
					3 & 1 & 1 & 4 \\
					\lambda & 4 & 10 & 1 \\
					1 & 7 & 17 & 3 \\
					2 & 2 & 4 & 3 \\
				\end{pNiceMatrix}.
			\end{equation*}
			\subsection*{Решение}
			Линейными преобразованиями сведем матрицу $\vec A$ к верхнетреугольной матрице с единицами на диагонали (для строки с параметром  $\lambda$ нельзя выполнять элементарные преобразования, которые могут сократить параметр или возвести его в степень):
			\begin{gather*}
				\begin{pNiceMatrix}
					3 & 1 & 1 & 4 \\
					\lambda & 4 & 10 & 1 \\
					1 & 7 & 17 & 3 \\
					2 & 2 & 4 & 3 \\
				\end{pNiceMatrix} 
				\rightarrow
				\begin{pNiceMatrix}
					1 & 1 & 2 & 1.5 \\
					0 & -2 & -5 & -0.5 \\
					0 & 6 & 15 & 1.5 \\					
					0 & 4 - \lambda & 10 - 2 \lambda & 1 - 1.5 \lambda \\		
				\end{pNiceMatrix}	
				\rightarrow	\\
				\rightarrow	
				\begin{pNiceMatrix}
					1 & 1 & 2 & 1.5 \\
					0 & 1 & 2.5 & 0.25 \\
					0 & 0 & 0 & 0 \\					
					0 & 0 & 10 - 2.5 (4 - \lambda) - 2 \lambda & 1 - 0.25 (4 - \lambda)- 1.5 \lambda \\		
				\end{pNiceMatrix}
				\rightarrow
				\begin{pNiceMatrix}
					1 & 1 & 2 & 1.5 \\
					0 & 1 & 2.5 & 0.25 \\
					0 & 0 & \dfrac{\lambda}{2} & -\dfrac{5 \lambda}{4} \\
					0 & 0 & 0 & 0 \\		
				\end{pNiceMatrix}.
			\end{gather*}
			Следовательно, если $\lambda = 0$, то $\mathrm{rank} (\vec A) = 2$, в противном случае
			$\mathrm{rank} (\vec A)= 3$.
				
	
	\newpage
	\section{Задача №4}
		\subsection*{Условие}
		Исследовать систему и найти решение в зависимости от значения параметра:
		\begin{equation*}
			\begin{cases}
				x_1 + 4 x_2 + 2 x_3 = -1, \\
				2 x_1 + 3 x_2 -1 x_3 = 3, \\
				x_1 - x_2 -3  x_3 = 4, \\
				x_1 - 6 x_2 - \lambda x_3 = 9, \\				
			\end{cases}
		\end{equation*}
		
		\subsection*{Решение}
		Перепишем в матричном виде $\vec A \vec x = \vec b$:
		\begin{equation*}
			\vec A = 
			\begin{pNiceMatrix}
				1 & 4 & 2  \\
				2 & 3 & -1  \\
				1 & -1 & -3 \\
				1 & -6 & - \lambda \\		
			\end{pNiceMatrix}, 
			\quad
			\vec x = 
			\begin{pNiceMatrix}
				x_1 \\
				x_2 \\
				x_3 \\
			\end{pNiceMatrix},
			\quad 
			\vec b = 
			\begin{pNiceMatrix}
				-1 \\
				3 \\
				4 \\
				9\\
			\end{pNiceMatrix}.		
		\end{equation*}
		Составим совместную матрицу $(\vec A | \vec b)$:
		\begin{equation*}
			\begin{pNiceArray}{ccc|c}
				1 & 4 & 2 & -1 \\
				2 & 3 & -1 & 3 \\
				1 & -1 & -3 & 4\\
				1 & -6 & - \lambda & 9 \\
			\end{pNiceArray}
			\rightarrow
			\begin{pNiceArray}{ccc|c}
				1 & -1 & -3 & 4\\
				0 & 5 & 5 & -5 \\
				0 & 5 & 5 & -5 \\
				0 & -5 & 3 - \lambda & 5 \\
			\end{pNiceArray}
			\rightarrow
			\begin{pNiceArray}{ccc|c}
				1 & -1 & -3 & 4\\
				0 & 1 & 1 & -1 \\
				0 & 0 & 8 - \lambda & 0 \\
				0 & 0 & 0 & 0 \\
			\end{pNiceArray}
		\end{equation*}
		Если $\lambda \ne 8$, то решение следующее: 
		\vspace*{-5mm}
		$$x_1 = 3,\ x_2 = -1,\ x_3 = 0 
		\quad \text{или} \quad
		\vec x = 
		\begin{pNiceMatrix}
			3 \\
			-1 \\
			0 \\
		\end{pNiceMatrix}.
		$$
		Рассмотрим случай, когда $\lambda = 8$. Продолжим преобразовывать систему: 
		\begin{equation*}
			\begin{pNiceArray}{ccc|c}
				1 & -1 & -3 & 4\\
				0 & 1 & 1 & -1 \\
				0 & 0 & 0 & 0 \\
				0 & 0 & 0 & 0 \\
			\end{pNiceArray}
			\rightarrow
			\begin{pNiceArray}{ccc|c}
				1 & 0 & -2 & 3\\
				0 & 1 & 1 & -1 \\
				0 & 0 & 0 & 0 \\
				0 & 0 & 0 & 0 \\
			\end{pNiceArray},
		\end{equation*}
		тогда решение имеет вид
		\begin{equation*}
			\begin{cases}
				x_1 = 3 + 2 x_3,\\
				x_2 = -1 - x_3, \\
				x_3 = x_3.
			\end{cases}
		\end{equation*}
	
		
	\newpage
	\section{Задача №5}
		\subsection*{Условие}
		Найти собственные значения и вектора для данной матрицы:
		\begin{equation*}
			\vec A =
			\begin{pNiceArray}{ccc}
				1 & -3 & 4\\
				4 & -7 & 8 \\
				6 & -7 & 7 \\
			\end{pNiceArray},
		\end{equation*}
		
		\subsection*{Решение}	
		Найдем корни характеристического уравнения
		$
			\left| \vec A - \lambda \vec E \right| = 0,
		$
		где $\vec E$ --- единичная матрица, $\lambda$ --- параметр, относительно которого решается уравнение:
		\begin{equation*}
			\left| \vec A - \lambda \vec E \right| = 
			\begin{vNiceMatrix}
				1 - \lambda & -3 & 4 \\
				4 & -7 - \lambda & 8 \\
				6 & -7 & 7 - \lambda \\
			\end{vNiceMatrix} = 3 + 5 \lambda + \lambda^2 - \lambda^3 = 0,
		\end{equation*}
		получим, что собственные значения для $\vec A$ следующие:
		\vspace*{-2.5mm}
		$$
		\lambda_{1,2} = -1, \quad \lambda_3 = 3.
		$$
		Найдем собственные векторы $\vec v_i, \ i = \overline{1, 3}$: 
		\begin{enumerate}
			\item $\lambda_{1,2} = -1$:
			\begin{gather*}
				\vec A - (-1) \vec E =
				\begin{pNiceMatrix}
					2 & -3 & 4\\
					4 & -6 & 8 \\
					6 & -7 & 8 \\
				\end{pNiceMatrix}
				\rightarrow
				\begin{pNiceMatrix}
					1 & -1.5 & 2\\
					0 & 1 & -2 \\
					0 & 0 & 0 \\
				\end{pNiceMatrix}
				\rightarrow
				\begin{pNiceMatrix}
					1 & 0 & -1 \\
					0 & 1 & -2 \\
					0 & 0 & 0 \\
				\end{pNiceMatrix}.
			\end{gather*}	
			Получаем, что
			$
			\vec v_{1,2} = 
			\begin{pNiceMatrix}
				x_3 \\
				2\, x_3\\
				x_3
			\end{pNiceMatrix},
			$	
			пусть $x_3 = 1$, тогда
			$
			\vec v_{1,2} = 
			\begin{pNiceMatrix}
				1 \\
				2 \\
				1
			\end{pNiceMatrix}.
			$	
			
			\item $\lambda_{3} = 3$:
			\begin{gather*}
				\vec A - (3) \vec E =
				\begin{pNiceMatrix}
					-2 & -3 & 4\\
					4 & -10 & 8 \\
					6 & -7 & 4\\
				\end{pNiceMatrix}
				\rightarrow
				\begin{pNiceMatrix}
					1 & 1.5 & -2\\
					0 & -16 & 16 \\
					0 & -16 & 16 \\
				\end{pNiceMatrix}
				\rightarrow
				\begin{pNiceMatrix}
					1 & 0 & -0.5 \\
					0 & 1 & -1 \\
					0 & 0 & 0 \\
				\end{pNiceMatrix}.
			\end{gather*}	
			Получаем, что
			$
			\vec v_{3} = 
			\begin{pNiceMatrix}
				0.5 \, x_3 \\
				x_3\\
				x_3
			\end{pNiceMatrix},
			$	
			пусть $x_3 = 2$, тогда
			$
			\vec v_{3} = 
			\begin{pNiceMatrix}
				1 \\
				2 \\
				2
			\end{pNiceMatrix}.
			$	
		\end{enumerate}
	\newpage
	\section{Задача №6}
		\subsection*{Условие}
			Произвести сингулярное разложение матрицы $\vec A$:
			\begin{equation*}
				\vec A = 
				\begin{pNiceMatrix}
					1 & -1 & -2 \\
					-7/3 & 1/3 & 2/3\\
					1/3 & -7/3 & -2/3\\
					-5/3 & 5/3 & -2/3\\
				\end{pNiceMatrix}.		
			\end{equation*}
		\subsection*{Решение}
		Цель получить разложение $\vec A$ такое, что $\vec A = \vec U \vec \Sigma \vec V^T$
		Найдем матрицу $\vec A \vec A^T$:
		\begin{equation*}
			\vec A \vec A^T = 
			\begin{pNiceMatrix}
				6 & -4 & 4 & 2 \\
				-4 & 6 & -2 & 4 \\
				4 & -2 & 6 & -4 \\
				-2 & 4 & -4 & 6 \\
			\end{pNiceMatrix}.
		\end{equation*}
		Найдем корни $\lambda$ уравнения $\left| \vec A \vec A^T - \lambda \vec E \right| = 0$, они будут следующие: 
		\vspace*{-2.5mm}
		$$\lambda_1 = 16, \ \lambda_{2,3} = 4, \ \lambda_4 = 0.$$
		Собственные векторы, соответствующие им: 
		$$
			\vec v_1 = 
			\begin{pNiceMatrix}
				-1\\
				1\\
				-1\\
				1
			\end{pNiceMatrix}, 
			\quad
			\vec v_2 = 
			\begin{pNiceMatrix}
				0\\
				1\\
				1\\
				0
			\end{pNiceMatrix}, 
			\quad	
			\vec v_3 = 
			\begin{pNiceMatrix}
				1\\
				0\\
				0\\
				1
			\end{pNiceMatrix}, 
			\quad	
			\vec v_4 = 
			\begin{pNiceMatrix}
				-1\\
				-1\\
				1\\
				1
			\end{pNiceMatrix}.
		$$
		Следовательно, сингулярные значения (корни не нулевых собственных значений матрицы $\vec A \vec A^T$) будут иметь вид
		\begin{equation*}
			\label{singular_vals}
			\sigma_1 = 4, \ \sigma_{2,3} = 2.
		\end{equation*}
		Пусть столбцы матрицы $\vec U$ --- нормированные собственные векторы $\vec A \vec A^T$:
		$$
			\vec U = 
			\begin{pNiceMatrix}
				-1/2 & 0 & 1/\sqrt{2} & -1/2\\
				1/2 & 1/\sqrt{2} & 0 & -1/2\\
				-1/2 & 1/\sqrt{2} & 0 & 1/2\\
				1/2 & 0 & 1/\sqrt{2} & 1/2
			\end{pNiceMatrix},
		$$
		и тогда (для сохранения верного размера итоговой матрицы)
		$$
		\vec \Sigma = 
		\begin{pNiceMatrix}
			4 & 0 & 0 \\
			0 & 2 & 0 \\
			0 & 0 & 2 \\
			0 & 0 & 0 \\
		\end{pNiceMatrix}.
		$$
		Так как матрица $\vec U$ ортогональная, то верно следующее:
		$$
			\vec V = (\vec \Sigma^{-1} \vec U^T \vec A)^T, 
			\quad
			\vec \Sigma^{-1} = 
			\begin{pNiceMatrix}
				1/4 & 0 & 0 & 0\\
				0 & 1/2 & 0 & 0 \\
				0 & 0 & 1/2 & 0 \\
			\end{pNiceMatrix}.
		$$
		После вычислений получим, что 
		$$
			\vec V = 
			\begin{pNiceMatrix}
				-2/3 & -1/\sqrt{2} & -\dfrac{\sqrt{2}}{6}\\
				2/3 & -1/\sqrt{2} & \dfrac{\sqrt{2}}{6} \\
				1/3 & 0 & -\dfrac{2 \sqrt{2}}{3}
			\end{pNiceMatrix}.
		$$
		
		Ответ:
		\begin{gather*}
			\vec A = \vec U \vec \Sigma \vec V^T, \quad \text{где}\\
			\vec U = 
			\begin{pNiceMatrix}
				-1/2 & 0 & 1/\sqrt{2} & -1/2\\
				1/2 & 1/\sqrt{2} & 0 & -1/2\\
				-1/2 & 1/\sqrt{2} & 0 & 1/2\\
				1/2 & 0 & 1/\sqrt{2} & 1/2
			\end{pNiceMatrix}, 
			\quad
			\vec \Sigma = 
			\begin{pNiceMatrix}
				4 & 0 & 0 \\
				0 & 2 & 0 \\
				0 & 0 & 2 \\
				0 & 0 & 0 \\
			\end{pNiceMatrix},	
			\quad
			\vec V = 
			\begin{pNiceMatrix}
				-2/3 & -1/\sqrt{2} & -\dfrac{\sqrt{2}}{6}\\
				2/3 & -1/\sqrt{2} & \dfrac{\sqrt{2}}{6} \\
				1/3 & 0 & -\dfrac{2 \sqrt{2}}{3}
			\end{pNiceMatrix}.	
		\end{gather*}
		
	

	
\end{document}